% arara: xelatex
\documentclass[11pt, letterpaper]{article}
\author{Ayberk Tosun}
\title{Notes on Cut Elimination}

\usepackage[margin=1in]{geometry}
\usepackage{mathpazo}
\linespread{1.05}
\usepackage[dvipsnames]{xcolor}
\usepackage{ebproof}
\usepackage{xparse}
\usepackage[euler-digits]{eulervm}

\definecolor{red}{RGB}{227, 56, 61}
\definecolor{blue}{RGB}{19, 80, 177}

\NewDocumentCommand \sequent { o m }{
  \IfNoValueTF{#1}
    { \Gamma      \Longrightarrow {#2}  }
    { \Gamma , {#1} \Longrightarrow {#2} }
}

\NewDocumentCommand \sequentHat { o m }{
  \IfNoValueTF{#1}
    { \widehat{\Gamma}      \Longrightarrow {\color{blue} #2}  }
    { \widehat{\Gamma} , {\color{blue} #1} \Longrightarrow {\color{blue} #2} }
}

\usepackage{listings}
\lstset{
  basicstyle=\footnotesize\ttfamily
}

\input{../../latex-basis/basis.tex}

\newcommand{\leftRule}[1]{\color{ForestGreen} #1}
\newcommand{\rightRule}[1]{\color{BurntOrange} #1}
\newcommand{\initRule}{\color{Plum} \mathsf{init}}

\newcommand{\ruleName}[1]{{\color{blue} #1}}
\newcommand{\mkLabel}[1]{\RightLabel{\ruleName{#1}}}

\newcommand{\conjIntro}{\ruleName{\times I}}
\newcommand{\conjElim}{\ruleName{\times E}}

\newcommand{\assert}[1]{{\color{Sepia} #1}\ {\color{gray} true}}
\newcommand{\prf}[1]{{\color{red} \mathcal{#1}}}
\newcommand{\prp}[1]{{\color{blue} #1}}
\newcommand{\subst}[3]{[#1 / \ruleName{#2}] #3}

\newcommand{\reduction}{\Longrightarrow_R}
\newcommand{\expansion}{\Longrightarrow_E}


\begin{document}
\maketitle

\section{Admissibility of Cut}

\begin{theorem}[Cut]
  If $\sequent{A}$ and $\sequent[A]{C}$ then $\sequent{C}$.
\end{theorem}
\begin{proof}
  Suppose that we have the derivations
  \begin{enumerate}
    \item $\prf{D}$ for $\sequent{A}$ and
    \item $\prf{E}$ for $\sequent[A]{C}$.
  \end{enumerate}
  We need to construct a derivation $\prf{F}$ for $\sequent{}{C}$.
  We consider the following six cases.

  \paragraph{Case: $\prf{D}$ is an initial sequent.} In this case
  $\prp{A}$ must be some atomic proposition $\prp{P}$ and $\prf{D}$
  must have the form
  \[
      \begin{prooftree}
        \infer0[init]{\sequentHat[P]{P}}
      \end{prooftree}
  \]
  So $\Gamma = (\widehat{\Gamma}, \prp{P})$ which implies that $\prf{E}$ proves
  $\sequentHat[\prp{P}, \prp{P}]{C}$. From this, it follows by contraction
  that $\sequentHat[\prp{P}]{C}$ and then since $\Gamma = (\Gamma, \prp{P})$,
  it is the case that $\sequent{\prp{C}}$.

  \paragraph{Case: $\prf{E}$ is an initial sequent that uses the cut formula.}
  Foo
  \todo[color=LimeGreen, inline]{\textsc{Complete case.}}

  \paragraph{Case: $\prf{E}$ is an initial sequent that does not use the cut
  formula.} Foo

  \todo[inline, color=LimeGreen]{\textsc{Complete case.}}

\end{proof}
\end{document}
