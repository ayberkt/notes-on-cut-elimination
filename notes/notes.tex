% arara: xelatex
% arara: bibtex
% arara: xelatex
% arara: xelatex
\documentclass[11pt, letterpaper]{article}
\author{Ayberk Tosun}
\title{Notes on Cut Elimination}

\usepackage[margin=1in]{geometry}
\usepackage{mathpazo}
\linespread{1.05}
\usepackage[dvipsnames]{xcolor}
\usepackage{ebproof}
\usepackage{xparse}
\usepackage[euler-digits]{eulervm}

\definecolor{red}{RGB}{227, 56, 61}
\definecolor{blue}{RGB}{19, 80, 177}

\NewDocumentCommand \sequent { o m }{
  \IfNoValueTF{#1}
    { \Gamma      \Longrightarrow {#2}  }
    { \Gamma , {\color{blue} #1} \Longrightarrow {#2} }
}

\NewDocumentCommand \sequentHat { o m }{
  \IfNoValueTF{#1}
    { \widehat{\Gamma}      \Longrightarrow {\color{blue} #2}  }
    { \widehat{\Gamma} , {\color{blue} #1} \Longrightarrow {\color{blue} #2} }
}

\usepackage{listings}
\lstset{
  basicstyle=\footnotesize\ttfamily
}

\input{../../latex-basis/basis.tex}

\newcommand{\ruleName}[1]{{\color{blue} #1}}
\newcommand{\mkLabel}[1]{\RightLabel{\ruleName{#1}}}

\newcommand{\conjIntro}{\ruleName{\times I}}
\newcommand{\conjElim}{\ruleName{\times E}}

\newcommand{\assert}[1]{{\color{Sepia} #1}\ {\color{gray} true}}
\newcommand{\prf}[1]{{\color{red} \mathcal{#1}}}
\newcommand{\prp}[1]{{\color{blue} #1}}
\newcommand{\subst}[3]{[#1 / \ruleName{#2}] #3}

\newcommand{\reduction}{\Longrightarrow_R}
\newcommand{\expansion}{\Longrightarrow_E}


\begin{document}
\maketitle

\section{Preliminaries}

Sequent calculus was conceived by Gentzen as a convenient intermediary language
for proving the cut-elimination theorem, which Girard \cite[pg.
106]{locus-solum} goes as far to deem ``the main achievement of Gentzen''.
Girard writes the following in his \emph{locus solum}
dictionary \cite[pg. 106]{locus-solum}:
\begin{quote}
  The cut-elimination theorem, which basically yields an algorithm to eliminate
  cut is surprising, since it relates two absolutely opposite views of logic.
  The deductive, or \emph{implicit} a pproach is about lemmas which contain
  main ideas that one combines together like modules in programming:
  a real proof is basically the cut rule. The \emph{explicit} approach is
  without cut, and in practice is never used, but by computers, since all
  ideas, i.e., lemmas, have disappeared. But a computer an idiot with big
  brains, can use cut-elimination, either to transform an implicit proof into
  an explicit one, or simply to look for a cut-free proof, since the absence
  of cut restricts proof-search to subformulas. Cut-elimination is a
  completeness theorem (the real one indeed), saying that the explicit
  approach is as strong as the implicit one.
\end{quote}

Even setting aside the proof of cut-elimination, sequent calculus exposes the
fundamental symmetries of logic, which makes it worth studying in itself.

Our sequent calculus is based on the following inference rules.

\[
  \begin{prooftree}
    \hypo{\sequent{A}}
    \hypo{\sequent{B}}
    \infer2[$\rightRule{\land R}$]{\sequent{A \land B}}
  \end{prooftree}
  \quad
  \begin{prooftree}
    \hypo{\sequent[A \land B][A]{C}}
    \infer1[$\leftRule{\land L_1}$]{\sequent[A \land B]{C}}
  \end{prooftree}
  \quad
  \quad
  \begin{prooftree}
    \hypo{\sequent[A \land B][B]{C}}
    \infer1[$\leftRule{\land L_2}$]{\sequent[A \land B]{C}}
  \end{prooftree}
\]
\vspace{0.5em}
\[
  \begin{prooftree}
    \infer0[$\rightRule{\top R}$]{\sequent{\top}}
  \end{prooftree}
\]
\vspace{0.5em}
\[
  \begin{prooftree}
    \hypo{\sequent[A \lor B, A]{C}}
    \hypo{\sequent[A \lor B, B]{C}}
    \infer2[$\leftRule{\lor L}$]{\sequent[A \lor B]{C}}
  \end{prooftree}
  \quad
  \begin{prooftree}
    \hypo{\sequent{A}}
    \infer1[$\rightRule{\vee R_1}$]{\sequent{A \lor B}}
  \end{prooftree}
  \quad
  \begin{prooftree}
    \hypo{\sequent{B}}
    \infer1[$\rightRule{\vee R_2}$]{\sequent{A \lor B}}
  \end{prooftree}
\]
\vspace{0.5em}
\[
  \begin{prooftree}
    \hypo{\sequent[A]{B}}
    \infer1[$\rightRule{\supset R}$]{\sequent{A \supset B}}
  \end{prooftree}
  \quad
  \begin{prooftree}
    \hypo{\sequent[A \supset B]{A}}
    \hypo{\sequent[A \supset B, B]{C}}
    \infer2[$\leftRule{\supset L}$]{\sequent[A \supset B]{C}}
  \end{prooftree}
\]
\vspace{0.5em}
\[
  \begin{prooftree}
    \infer0[$\leftRule{\bot L}$]{\sequent[\bot]{C}}
  \end{prooftree}
\]

In addition to these, we have the $\initRule$ rule to verify an atomic formula
which can only be verified if it is already on the left.
\[
  \begin{prooftree}
    \infer0[$\initRule$]{\sequent[P]{P}}
  \end{prooftree}
\]

\section{Admissibility of Cut}

\begin{theorem}[Cut]
  If $\sequent{A}$ and $\sequent[A]{C}$ then $\sequent{C}$.
\end{theorem}
\begin{proof}
  Suppose that we have the derivations
  \begin{enumerate}
    \item $\drv{D}$ for $\sequent{A}$ and
    \item $\drv{E}$ for $\sequent[A]{C}$.
  \end{enumerate}
  We need to construct a derivation $\drv{F}$ for $\sequent{}{C}$.
  We consider the following cases.

  \paragraph{Case: $\drv{D}$ is an initial sequent.} In this case
  $\prp{A}$ must be some atomic proposition $\prp{P}$ and $\drv{D}$
  must have the form
  \[
      \begin{prooftree}
        \infer0[$\initRule$]{\sequentHat[P]{P}}
      \end{prooftree}
  \]
  So $\Gamma = (\widehat{\Gamma}, \prp{P})$ which implies that $\drv{E}$ is a
  derivation of $\sequentHat[\prp{P}, \prp{P}]{C}$. From this, it follows by
  contraction that $\sequentHat[\prp{P}]{C}$ and then since $\Gamma = (\Gamma,
  \prp{P})$, it is the case that $\sequent{\prp{C}}$.

  \paragraph{Case: $\drv{E}$ is an initial sequent.}

  \begin{itemize}
    \item \textbf{Subcase: the initial sequent uses the cut formula.} So it
    must be the case that $\prp{A} = \prp{C} = \prp{P}$. Hence $\drv{E}$ has
    the form
    \[
      \begin{prooftree}
        \infer0[$\initRule$.]{\sequent[P]{P}}
      \end{prooftree}
    \]
    This means that $\drv{D}$ is a derivation of $\sequent{P}$ and hence
    $\sequent{C}$.

    \item \textbf{Subcase: the initial sequent does not use the cut formula.}
    In this case $\prp{C} = \prp{P}$ where $\prp{P}$ is some atomic formula,
    $\Gamma = (\widehat{\Gamma}, \prp{P})$ and $\drv{E}$ has the form:
    \[
      \begin{prooftree}
        \infer0[$\initRule$]{\sequentHat[\prp{P}, \prp{A}]{\prp{P}}}
      \end{prooftree}
    \]
    As the $\initRule$ rule does not make use of $\prp{A}$ we may take that out
    and the derivation would remain valid for $\sequentHat[\prp{P}]{\prp{P}}$.
    Hence we have $\sequent{\prp{P}}$ i.e., $\sequent{\prp{C}}$.
  \end{itemize}

  \paragraph{Case: $\prp{A}$ is the principal formula of the final inference
  in both $\drv{D}$ and $\drv{E}$.}
  \begin{itemize}
    \item \textbf{Subcase: $\prp{A}$ has the form $\prp{A_1 \land A_2}$.}
    This implies that $\drv{D}$ uses $\rightRule{\land R}$ in its last step:
    \[
      \begin{prooftree}
        \hypo{{\color{Red} \mathcal{D}_1}}
        \infer[no rule]1{\sequent{A_1}}
        \hypo{{\color{Red} \mathcal{D}_2}}
        \infer[no rule]1{\sequent{A_2}}
        \infer2[$\rightRule{\land R}$,]{\sequent{A_1 \land A_2}}
      \end{prooftree}
    \]
    whereas $\drv{E}$ has the form:
    \[
      \begin{prooftree}
        \hypo{{\color{Red} \mathcal{E}_1}}
        \infer[no rule]1{\sequent[A_1 \land A_2][A_1]{C}}
        \infer1[$\leftRule{\land L}$.]{\sequent[A_1 \land A_2]{C}}
      \end{prooftree}
    \]
    To be able to make use of the induction hypothesis, we have to weaken
    $\drv{D}$. We may do this without changing the structure of $\drv{D}$ which
    is what we are interested in; let us call this $\drv{\widehat{D}}$:
    \[
      \begin{prooftree}
        \hypo{\drv{\widehat{D}}[1]}
        \infer[no rule]1{\sequent[A_1]{A_1}}
        \hypo{\drv{\widehat{D}}[2]}
        \infer[no rule]1{\sequent[A_1]{A_2}}
        \infer2[$\rightRule{\land R}$.]{\sequent[A_1]{A_1 \land A_2}}
      \end{prooftree}
    \]
    By the IH on $\prp{A_1 \land A_2}$, $\drv{D}$, and
    ${\color{Red} \mathcal{E}_1}$ we may obtain some evidence $\drv{G}$
    of $\sequent[A_1]{C}$. By the IH, again, on $\prp{A_1}$,
    $\drv{\widehat{D}}[1]$, and $\drv{G}$, we get $\sequent{C}$.
    % \item \textbf{Subcase: $\prp{A} = \prp{\top}$.}
    % $\prp{\top}$ has only the right rule $\rightRule{\top R}$. So $\drv{D}$
    % must have the form:
    % \[
    %   \begin{prooftree}
    %     \infer0[$\rightRule{\top R}$.]{\sequent{\top}}
    %   \end{prooftree}
    % \]
    % Since there are no left rules for $\prp{\top}$, for any $\Gamma$ with
    % $\sequent[\top]{C}$ it will also be the case that $\sequent{C}$. In
    % other words, $\drv{E}$ must have the form:
    % \[
    %   \begin{prooftree}
    %     \hypo{\drv{E}[1]}
    %     \infer[no rule]1{\sequent{C}}
    %     \infer1[,]{\sequent[\top]{C}}
    %   \end{prooftree}
    % \]
    % in which $\drv{E}[1]$ constitutes evidence for $\sequent{C}$.

    \item \textbf{Subcase: $\prp{A}$ has the form $\prp{A_1 \lor A_2}$.}
    In this case, $\drv{E}$ must be using either $\rightRule{\lor R_1}$
    or $\rightRule{\lor R_2}$. If it is using $\rightRule{\lor R_1}$, then
    $\drv{D}$ has the form:
    \[
      \begin{prooftree}
        \hypo{\drv{D}[1]}
        \infer[no rule]1{\sequent{A_1}}
        \infer1[$\rightRule{\lor R_1}$]{\sequent{A_1 \lor A_2}}
      \end{prooftree}
    \]
    whereas $\drv{E}$ uses $\leftRule{\lor L}$ in its last step:
    \[
      \begin{prooftree}
        \hypo{\drv{E}[1]}
        \infer[no rule]1{\sequent[A_1 \lor A_2][A_1]{C}}
        \hypo{\drv{E}[2]}
        \infer[no rule]1{\sequent[A_1 \lor A_2][A_2]{C}}
        \infer2[$\leftRule{\lor L}$.]{\sequent[A_1 \lor A_2]{C}}
      \end{prooftree}
    \]
    Note that we may weaken $\drv{D}$ with $\prp{A_1}$ to obtain a derivation
    $\drv{\widehat{D}}$ that looks like:
    \[
      \begin{prooftree}
        \hypo{\drv{\widehat{D}}[1]}
        \infer[no rule]1{\sequent[A_1]{A_1}}
        \infer1[$\rightRule{\lor R_1}$.]{\sequent[A_1]{A_1 \lor A_2}}
      \end{prooftree}
    \]
    Now by using the IH on $\prp{A_1 \lor A_2}$, $\drv{\widehat{D}}$, and,
    $\drv{E}[1]$ we get $\sequent[A]{C}$.

    If $\drv{D}$ uses $\rightRule{\lor R_1}$, then we weaken it with
    $\prp{A_2}$ and appeal to IH with $\drv{E}[2]$ instead.

    \item \textbf{Subcase: $\prp{A}$ has the form $\prp{A_1 \supset A_2}$.}
    This is to say that $\drv{D}$ looks like:
    \[
      \begin{prooftree}
        \hypo{\sequent[A_1]{A_2}}
        \infer1[$\rightRule{\supset R}$]{\sequent{A \supset A_2}}
      \end{prooftree}
    \]
  \end{itemize}

  \paragraph{Case: $\prp{A}$ is not the principal formula of the last
  inference in $\drv{D}$.}\quad
  \todo[inline, color=LimeGreen]{Complete.}

  \paragraph{Case: $\prp{A}$ is not the principal formula of the last
  inference in $\drv{E}$.}\quad
  \todo[inline, color=LimeGreen]{Complete.}

\end{proof}

\bibliographystyle{alpha}
\bibliography{bibliography}
\end{document}
