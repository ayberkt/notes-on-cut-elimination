% arara: xelatex
% arara: bibtex
% arara: xelatex
% arara: xelatex
\documentclass[11pt, letterpaper]{article}
\author{Ayberk Tosun}
\title{Notes on Cut Elimination}

\usepackage[margin=1in]{geometry}
\usepackage{mathpazo}
\linespread{1.05}
\usepackage[dvipsnames]{xcolor}
\usepackage{ebproof}
\usepackage{xparse}
\usepackage[euler-digits]{eulervm}

\definecolor{red}{RGB}{227, 56, 61}
\definecolor{blue}{RGB}{19, 80, 177}

\NewDocumentCommand \sequent { o m }{
  \IfNoValueTF{#1}
    { \Gamma      \Longrightarrow {#2}  }
    { \Gamma , {#1} \Longrightarrow {#2} }
}

\NewDocumentCommand \sequentHat { o m }{
  \IfNoValueTF{#1}
    { \widehat{\Gamma}      \Longrightarrow {\color{blue} #2}  }
    { \widehat{\Gamma} , {\color{blue} #1} \Longrightarrow {\color{blue} #2} }
}

\usepackage{listings}
\lstset{
  basicstyle=\footnotesize\ttfamily
}

\input{../../latex-basis/basis.tex}

\newcommand{\leftRule}[1]{\color{ForestGreen} #1}
\newcommand{\rightRule}[1]{\color{BurntOrange} #1}
\newcommand{\initRule}{\color{Plum} \mathsf{init}}

\newcommand{\ruleName}[1]{{\color{blue} #1}}
\newcommand{\mkLabel}[1]{\RightLabel{\ruleName{#1}}}

\newcommand{\conjIntro}{\ruleName{\times I}}
\newcommand{\conjElim}{\ruleName{\times E}}

\newcommand{\assert}[1]{{\color{Sepia} #1}\ {\color{gray} true}}
\newcommand{\prf}[1]{{\color{red} \mathcal{#1}}}
\newcommand{\prp}[1]{{\color{blue} #1}}
\newcommand{\subst}[3]{[#1 / \ruleName{#2}] #3}

\newcommand{\reduction}{\Longrightarrow_R}
\newcommand{\expansion}{\Longrightarrow_E}


\begin{document}
\maketitle

\section{Preliminaries}

Sequent calculus was conceived by Gentzen as a convenient intermediary language
for proving the cut-elimination theorem, which Girard \cite[pg.
106]{locus-solum} goes as far to deem ``the main achievement of Gentzen''.
Further in Girard's words \cite[pg. 106]{locus-solum}:
\begin{quote}
  The cut-elimination theorem, which basically yields an algorithm to eliminate
  cut is surprising, since it relates to absolutely opposite views of logic.
  The deductive, or \emph{implicit} a pproach is about lemmas which contain
  main ideas that one combines together like modules in programming:
  a real proof is basically the cut rule. The \emph{explicit} approach is
  without cut, and in practice is never used, but by computers, since all
  ideas, i.e., lemmas, have disappeared. But a computer an idiot with big
  brains, can use cut-elimination, either to transform an implicit proof into
  an explicit one, or simply to look for a cut-free proof, since the absence
  of cut restricts proof-search to subformulas. Cut-elimination is a
  completeness theorem (the real one indeed), saying that the explicit
  approach is as strong as the implicit one.
\end{quote}

Even setting aside the proof of cut-elimination, sequent calculus exposes the
fundamental symmetries of logic, which makes it worth studying in itself.

Our sequent calculus is based on the following inference rules.

\[
  \begin{prooftree}
    \hypo{\sequent{\prp{A}}}
    \hypo{\sequent{\prp{B}}}
    \infer2[$\rightRule{\land R}$]{\sequent{\prp{A \land B}}}
  \end{prooftree}
  \quad
  \begin{prooftree}
    \hypo{\sequent[\prp{A \land B}, \prp{A}]{\prp{C}}}
    \infer1[$\leftRule{\land L_1}$]{\sequent[\prp{A \land B}]{\prp{C}}}
  \end{prooftree}
  \quad
  \quad
  \begin{prooftree}
    \hypo{\sequent[\prp{A \land B}, \prp{B}]{\prp{C}}}
    \infer1[$\leftRule{\land L_2}$]{\sequent[\prp{A \land B}]{\prp{C}}}
  \end{prooftree}
\]
\vspace{0.5em}
\[
  \begin{prooftree}
    \infer0[\rightRule{\top R}]{\sequent{\prp{\top}}}
  \end{prooftree}
\]
\vspace{0.5em}
\[
  \begin{prooftree}
    \hypo{\sequent[\prp{A \lor B}, \prp{A}]{C}}
    \hypo{\sequent[\prp{A \lor B}, \prp{B}]{C}}
    \infer2[\leftRule{\lor L}]{\sequent[\prp{A \lor B}]{\prp{C}}}
  \end{prooftree}
  \quad
  \begin{prooftree}
    \hypo{\sequent{\prp{A}}}
    \infer1[$\rightRule{\vee R_1}$]{\sequent{\prp{A \lor B}}}
  \end{prooftree}
  \quad
  \begin{prooftree}
    \hypo{\sequent{\prp{B}}}
    \infer1[$\rightRule{\vee R_2}$]{\sequent{\prp{A \lor B}}}
  \end{prooftree}
\]
\vspace{0.5em}
\[
  \begin{prooftree}
    \hypo{\sequent[\prp{A}]{\prp{B}}}
    \infer1[\rightRule{\supset R}]{\sequent{A \supset B}}
  \end{prooftree}
  \quad
  \begin{prooftree}
    \hypo{\sequent[\prp{A \supset B}]{\prp{A}}}
    \hypo{\sequent[\prp{A \supset B}, \prp{B}]{C}}
    \infer2[\leftRule{\supset L}]{\sequent[\prp{A \supset B}]{\prp{C}}}
  \end{prooftree}
\]
\vspace{0.5em}
\[
  \begin{prooftree}
    \infer0[\leftRule{\bot L}]{\sequent[\prp{\bot}]{C}}
  \end{prooftree}
\]

In addition to these, we have the $\initRule$ rule to verify an atomic formula
which can only be verified if it is already on the left.
\[
  \begin{prooftree}
    \infer0[$\initRule$]{\sequent[\prp{P}]{\prp{P}}}
  \end{prooftree}
\]

\section{Admissibility of Cut}

\begin{theorem}[Cut]
  If $\sequent{A}$ and $\sequent[A]{C}$ then $\sequent{C}$.
\end{theorem}
\begin{proof}
  Suppose that we have the derivations
  \begin{enumerate}
    \item $\prf{D}$ for $\sequent{A}$ and
    \item $\prf{E}$ for $\sequent[A]{C}$.
  \end{enumerate}
  We need to construct a derivation $\prf{F}$ for $\sequent{}{C}$.
  We consider the following six cases.

  \paragraph{Case: $\prf{D}$ is an initial sequent.} In this case
  $\prp{A}$ must be some atomic proposition $\prp{P}$ and $\prf{D}$
  must have the form
  \[
      \begin{prooftree}
        \infer0[$\initRule$]{\sequentHat[P]{P}}
      \end{prooftree}
  \]
  So $\Gamma = (\widehat{\Gamma}, \prp{P})$ which implies that $\prf{E}$ proves
  $\sequentHat[\prp{P}, \prp{P}]{C}$. From this, it follows by contraction
  that $\sequentHat[\prp{P}]{C}$ and then since $\Gamma = (\Gamma, \prp{P})$,
  it is the case that $\sequent{\prp{C}}$.

  \paragraph{Case: $\prf{E}$ is an initial sequent that uses the cut formula.}
  \quad
  \todo[inline, color=LimeGreen]{Complete.}

  \paragraph{Case: $\prf{E}$ is an initial sequent that does not use the cut
  formula.}\quad\todo[inline, color=LimeGreen]{Complete}

  \paragraph{Case: $\prp{A}$ is the principal formula of the final inference
  in both $\prf{D}$ and $\prf{E}$.}\quad
  \todo[inline, color=LimeGreen]{Complete.}

  \paragraph{Case: $\prp{A}$ is not the principial formula of the last
  inference in $\prf{D}$.}\quad
  \todo[inline, color=LimeGreen]{Complete.}

  \paragraph{Case: $\prp{A}$ is not the principal formula of the last
  inference in $\prf{E}$.}\quad
  \todo[inline, color=LimeGreen]{Complete.}

\end{proof}

\bibliographystyle{alpha}
\bibliography{bibliography}
\end{document}
